\appendix
\chapter{Procuring the Hardware}\label{shield-appendix}
One may need some or all of the components listed below to carry out
the experiments explained in the book:
\begin{enumerate}
  \item Arduino Uno
  \item Components required to build the shield
  \item External actuators that work with the shield
  \item Components required for the breadboard based experiments
\end{enumerate}
We now provide an approximate cost of the above parts, mainly to keep
a beginner informed, see \tabref{tab:cost}.  Because of the
fluctuations in exchange rates, and the availability issues, pricing
could change considerably.

\begin{table}
  \centering
  \caption{Approximate cost of the components}
  \label{tab:cost}
  \begin{tabular}{|p{5cm}|c|}\hline
    Item                                                           & Cost (\rupee) \\ \hline
    Arduino Uno                                                    & 675           \\ \hline
    Components for the shield, given in \tabref{tab:shield-values} &
    350                                                                            \\ \hline
    External actuators (motor, motor driver, servo motor) to work with
    the shield                                                     & 400           \\ \hline
    Half breadboard                                                & 250           \\ \hline
    Components that go with the breadboard                         & 700           \\ \hline
  \end{tabular}
\end{table}

With the components and the Gerber file mentioned in
\secref{shield-hw}, it should be possible to create the PCB, and
solder the components, so as to arrive at the shield, as
shown in \figref{shield}.  This is the least expensive option.

% A lot more expensive option is to request the FOSSEE Team to create a
% shield and to supply it.  It is a lot more expensive, because the cost
% of the manpower who will make the shield is also to be factored.  As
% this work is not a part of IIT Bombay's regular activity, it has to be
% carried out through personnel specifically employed for this purpose.
% The FOSSEE team has a limited supply of shields, available at
% \rupee1,000 each.  Interested people may contact
% \href{mailto:FLOSS-arduino@fossee.in}{FLOSS-arduino@fossee.in} or at
% +91 22 2576 4133.  As mentioned above, the cheapest way is to make the
% shield by oneself, by procuring the components and making the PCB.

The readymade shield is also available on E-commerce websites like Amazon \cite{amazon-shield}
and Flipkart \cite{flipkart-shield}, in the name of Ecolight\textregistered \ Sensor Shield V-1.2
Compatible with Arduino Uno R3.

Interested people may contact
\href{mailto:FLOSS-arduino@fossee.in}{FLOSS-arduino@fossee.in} or at
+91 22 2576 4133.


% In addition to this, if you write to us at the above email address, 
% we may be in a position to give details of vendors who can supply the above components, and
% also the shield.  The advantage may be that these vendors may be able
% to supply these components and the shield at a price lower than that
% of FOSSEE. In case the readers know about some vendors who can meet
% this demand, please let us know and we will be happy to include them
% in our list.



