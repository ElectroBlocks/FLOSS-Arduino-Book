\appendix
\chapter{Procuring the Hardware}\label{shield-appendix}
One may need some or all of the components listed below to carry out
the experiments explained in the book:
\begin{enumerate}
\item Arduino Uno
\item Components required to build the shield
\item External actuators that work with the shield
\item Components required for the breadboard based experiments 
\end{enumerate}
We now provide an approximate cost of the above parts, mainly to keep
a beginner informed.  Because of the fluctuations in exchange rates,
and the availability issues, pricing could change considerably.

\begin{table}
  \centering
  \caption{Approximate cost of the components}
  \label{tab:cost}
  \begin{tabular}{|p{5cm}|c|}\hline
    Item & Cost (\rupee) \\ \hline
    Arduino Uno & 450 \\ \hline
    Components for the shield, given in \tabref{tab:shield-values} &
    300 \\ \hline
    External actuators (motor, motor driver, servo motor) to work with
    the shield & 400 \\ \hline
    Half breadboard & 250 \\ \hline
    Components that go with the breadboard & 700 \\ \hline
  \end{tabular}
\end{table}

All the
components mentioned above are available from 
\begin{quote}
  Aditya Enterprises \\
  Plot no. F-16, Rathi Sansar, Pisadevi Road \\ Aurangabad - 431001, Maharashtra. \\
  Email: admin@adityaentr.com $||$
  Contact: +91-9822094359
\end{quote}

With the components and the Gerber file mentioned in
\secref{shield-hw}, it should be possible to create the shield, as
shown in \figref{shield}.  This is the least expensive option.

The company mentioned above is also in a position to provide the
shield with all components soldered.  As already explained, the shield
is a printed circuit board (PCB) with a large number of sensors,
already wired and hence, ready to use. The shield provides the user a
faster way of circuit prototyping without worrying much about
troubleshooting.  

These components may be available from the E-commerce platforms such
as Amazon and Flipkart.  We will be happy to list other vendors who
can provide the components or the shield or both.  The FOSSEE team is,
however, not in a position to certify any of them.

The FOSSEE team has a limited supply of shields, available at
\rupee1,000 each.  Interested people may contact info@fossee.in or at
+91 22 2576 4133.  It should be mentioned that the cost of producing
the shield by FOSSEE is very high, owing to the manpower and the
infrastructure cost.  As a matter of fact, making hardware is not the
forte of FOSSEE.  As mentioned above, the cheapest way is to procure
the components and building the shield.

This price difference may provide a viable business model for
efficient vendors.

%% Even though the shield obviates the need for a breadboard as an
%% intermediate tool for electronics circuit prototyping, some learners
%% might prefer to use a breadboard and other required sensors to design
%% their circuits. That's why we have added breadboard connection
%% diagrams in the book, wherever required.
