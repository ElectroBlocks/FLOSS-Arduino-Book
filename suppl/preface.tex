\chapter*{Preface\markboth{\bf Preface}{}}
\thispagestyle{empty}
\addcontentsline{toc}{chapter}{\protect\numberline{Preface}}

Seeds for Oscad were sown when the National Mission on Education
through ICT (NMEICT) was launched: the mission document identified
\emph{Adaption \& deployment of open source simulation packages
  equivalent to Matlab, OrCAD, etc.}, as one of the areas NMEICT would
concentrate on.  
The FOSSEE (free and open source software in science and engineering
education) group at IIT Bombay, of which we are a part of, initially
started working on Python and Scilab.  The Standing Committee of
NMEICT encouraged us to contribute to other open source software as
well.  This push helped us develop Oscad, an open source alternative
to OrCAD.

Oscad is an electronic design automation (EDA) tool, developed using
KiCad, Ngspice and Scilab.  We have made the netlist files generated
by KiCad suitable for simulation through Ngspice.  In order to provide
an explanation facility, we have developed a method to automatically
generate differential equations that describe a given analog circuit
and to solve them using Scilab.  Once satisfied with simulation
results, the user can create a Gerber file for PCB fabrication.

While working on Scilab and Python, the FOSSEE group, jointly with the
Spoken Tutorial team, created a large number of Spoken Tutorials
\cite{kmm11-csi}.  Spoken Tutorials are audio-video tutorials in the
IT and simulation areas, created for self learning using screencast
technology.  This instructional material has been used to train more
than 20,000 college students on Scilab and Python in the past two
years.

We have created seven spoken tutorials of ten minutes each, using
which, a beginner level SELF workshop can be conducted on Oscad.  We
plan to conduct these workshops in about 100 colleges in the next one
year, free of cost.

The FOSSEE team has also created more than 160 Scilab Textbook
Companions, each of which contains Scilab code for worked out examples
of standard textbooks, mostly in engineering and science.  These have
been created by the students and professors from various
educational institutions in India.  These textbooks can be downloaded
free of cost from \cite{scilab}.  They can also be executed remotely
on GARUDA cloud \cite{GARUDA}.

We are embarking on a similar methodology for Oscad as well: we have
solved most of the worked out examples of \cite{sedra} and given the
solution in Appendix~\ref{ch:appen}.  We hope to create Oscad Textbook
Companions for all other relevant standard textbooks as well in the
near future, once again through students and other volunteers.

Solving the worked out examples of \cite{sedra} was a good exercise,
as it helped identify and
include some missing features.  The yet to be created Oscad Textbook
Companions are expected to help in this regard, while simultaneously
increasing the available documentation.

Lab migration is another important activity that the FOSSEE team is
involved in.  It provides equivalent Scilab code for Matlab based
labs.  This is also carried out through students and volunteers.  We
are starting this activity for Oscad as well: we will try to provide
equivalent Oscad based solution to all circuit design labs that
currently use proprietary software.

We have successfully ported Oscad on Aakash, the world's lowest cost
computing tablet.  As Ubuntu 12.10 runs on native mode on Aakash, we
could port Oscad to it.  \chapref{chap11} explains this activity,
along with a few screenshots.  As the Aakash tablet costs Rs. 2,263,
and hence, for less than Rs. 2,500 (including a keyboard and a mouse),
one can get access to a powerful EDA system.  This is expected to help
the students who are enthusiastic about circuit design, but cannot
afford expensive hardware and software.

Porting of Oscad demonstrates the power of the concept of Aakash: an
unlimited number of open source educational software systems can be
made available even in a low cost device.  Aakash can serve the dual
purpose of a tablet and a computing device.  This is the only way to
address the aspirations of the millions of poor students who cannot
afford even a computer system or an expensive tablet, let alone
both.

The FOSSEE team is currently working on the promotion/development of
the following open source software systems as well: 
\begin{inparaenum}
\item OpenFOAM, a CFD solver and an open source alternative to Fluent
  and StarCD.
\item COIN-OR, an open source software suite for optimisation
  problems. 
\item OpenFormal for formal verification of computer software.  
\end{inparaenum}
About ten professors and 25 full time staff members and students are
working on FOSSEE projects at IIT Bombay.  Many more are expected to
join in the near future.  

Another important project supported by NMEICT is the Teach 10,000
Teachers (T10KT) programme.  This methodology, pioneered at IIT Bombay
\cite{T10KT,T10KT-kal} has demonstrated that it is possible for the best people
in the field to provide extremely high quality training to a large
number of learners simultaneously.  Oscad is expected to be used in
the forthcoming T10KT course on Analog Electronics, organised by IIT
Kharagpur \cite{T10KT-kgp}.

We invite all EDA enthusiasts to work with us through the following
resources:
\begin{inparaenum}
\item URL for all FOSSEE activities: http://fossee.in
\item URL for all Oscad resources: http://oscad.in 
\item Textbook companion: textbook-companion@oscad.in
\item Lab migration: lab-migration@oscad.in
\item SELF workshops: SELF-workshop@oscad.in
\item Oscad development and enhancing its capabilities:
  Oscad-dev@oscad.in 
\item Feedback on this book: Oscad-textbook@oscad.in.
\end{inparaenum}
We also hope to establish forum based discussion services for
Oscad.  

Finally, an electronic version of this book is available for
noncommercial purposes at http://oscad.in.

\clearpage
\section*{Acknowledgements}
\addcontentsline{toc}{chapter}{\protect\numberline{Acknowledgements}}
We would first like to thank Mr. N. K. Sinha, IAS, for without him,
there would have been no National Mission on Education through ICT
(NMEICT), without which, there would have been no FOSSEE, without
which, there would have been no Oscad.  The idealistic guiding
principles of NMEICT, namely, reliance on open source software,
providing free access to e-content, Internet connectivity for all
educational institutions and providing a low cost access device to
every student through Aakash, egged us to contribute our best and one
of the outcomes is Oscad.

We would like to thank the former Human Resource Development Minister
(HRM) Mr. Arjun Singh for getting NMEICT started.  We would like to
acknowledge the former HRM Mr. Kapil Sibal for his unstinting support
and the faith he had in the NMEICT administration team.  We would like
to thank the current HRM Dr. Pallam Raju for extending the tenure of
NMEICT by five more years.

We want to thank the Members of the Standing Committee of NMEICT who
met once in two weeks for almost two years to review project proposals
and to recommend them for funding or giving suggestions for
improvement.  We also want to thank them for urging us to work on more
FOSS systems than what we were prepared for.  Without this kind of
active support, the ecosystem required for projects like Oscad to
flourish, established at IIT Bombay through the many projects funded
through NMEICT, would not have materialised.

We want to thank the FOSSEE faculty members Profs. Prabhu
Ramachandran, Madhu Belur, Mani Bhushan, Shiva Gopalakrishnan,
Jayendran Venkateswaran, Ashutosh Mahajan and Supratik Chakraborty for
establishing a vibrant FOSSEE group at IIT Bombay.  We want to thank
Prof. D. B. Phatak for being a constant source of inspiration and
encouragement and for supporting our activities directly and
indirectly through the Teach 10,000 Teacher Programme \cite{T10KT} and
the Aakash \cite{aakash} Project.  We want to thank other faculty
members with NMEICT projects at IIT Bombay, namely, Profs. Kavi Arya,
Ravi Poovaiah, Santosh Noronha, Anil Kulkarni, Sridhar Iyer, Sahana
Murthy and Shishir Jha for sharing their dreams, processes and
facilities.  We want to thank the staff members of all NMEICT projects
at IIT Bombay in general and of FOSSEE and Spoken Tutorial projects in
particular, for providing a wonderful work environment.

We want to thank the IIT Bombay administration in general and R\&D
office in particular for providing us with an excellent environment to
make us work efficiently.  We want to thank the researchers and
faculty members in our departments for providing us with necessary
space and for putting up with our tantrums.

We would like to thank the professors, staff and students affiliated
with the Wadhwani Electronics lab at IIT Bombay for trying out Oscad
in lab courses and for the useful suggestions.  We would like to thank
Abhishek Pawar  for creating Spoken Tutorials on KiCad.  We would like
to thank Saket Choudhary for making the netlist files generated by
KiCad compatible with Ngspice.  We want to thank Hardik for his help
in implementing the current GUI of Oscad.  We want to thank Kiran for
designing the logo of Oscad.  We want to thank Bella for helping with
the coordination of FOSSEE in general and Oscad in particular.  We
want to thank Mr.~Sunil Shastri of Shroff Publishers for
bringing out this book in a short time.

Finally, we want to thank our family members for allowing us to work
extended hours and for bearing with us. \\ [1cm]

\setlength{\tabcolsep}{0.5cm}
\begin{center}
\begin{tabular}{ccc}
Yogesh Save & Rakhi R & Shambhulingayya N. D. \\ [1mm]
Rupak M. Rokade & Ambikeshwar Srivastava & Manas Ranjan Das \\ [1mm] 
Lavitha Pereira & Sachin Patil & Srikant Patnaik \\ [1mm]
& Kannan M. Moudgalya \\ [5mm]
& IIT Bombay \\
& 22 May 2013
\end{tabular}
\end{center}

\cleardoublepage
