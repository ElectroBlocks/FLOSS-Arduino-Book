\chapter*{Preface}
\addcontentsline{toc}{chapter}{\protect\numberline{Preface}}
Microcontrollers are extensively used in the
industry, automobiles and home appliances, to list a few. 
Arduino is a popular open source microcontroller available today.
Arduino comes with its own software, Arduino IDE, short for Integrated
Development Environment.  Using Arduino IDE, one can program Arduino
for different purposes.  Some times, one may want to program Arduino
using high level programming languages.  We have explained Arduino
programming with the following high level languages through separate
books. 
\begin{center}
\begin{tabular}{llp{5.5cm}}
1 & Scilab and Xcos: & Microcontroller programming with Arduino,
Scilab, and Xcos \\
2 & Python: & Microcontroller programming with Arduino and Python \\
3 & OpenModelica: & Microcontroller programming with Arduino and
OpenModelica \\ 
4 & Julia: & Microcontroller programming with Arduino and Julia \\
\end{tabular}
\end{center}

Each of these four books concentrates on one topic.  Such a compact
book may be of interest to those who are interested in only one
programming language.  We are bringing out a fifth book that combines
all of these languages for the benefit of readers 
who may be interested in two or more languages.

All code explained in the book are made freely downloadable and their
URL are provided at appropriate places.  The user will need the
\arduino\ board and a Shield to carry out the experiments explained
in this book.  It is easy to procure \arduino\, which is available
through many vendors.  The location of the Gerber file and details of
the components required to assemble the Shield are provided in the
book.  Alternatively, one can buy the Shield directly from vendors,
details of whom are also provided in the book.

This work is supported by the FOSSEE (Free/Libre and Open Source
Software for Education) project (\url{https://fossee.in}) at IIT
Bombay.  This project is funded by the National Mission on Education
through ICT, Ministry of Education, Govt. of India.  FOSSEE promotes
software, such as Scilab, Python, eSim, OpenFOAM, OpenModelica, DWSIM,
Osdag, R, GIMP, Blender, Inkscape, ChemCollective Virtual Lab, and
Jmol.  It trains students on these software to the level they can
contribute, curates such contributions, and releases them to the
public.  It also conducts various competitions/hackathons/marathons,
handholds the motivated who need help, and recognises the best
contributions.  Students who have participated in FOSSEE projects have
benefited in terms of internship, employment, and admission to higher
studies with schloarship.

The FOSSEE project provides support to the Arduino activity through
the website \url{https://floss-arduino.fossee.in/}.  It has a lot of
useful information, such as Spoken Tutorials, and links to download
soft copies of these books.  We invite the learners to try out these
books and give their feedback.  \\ [0.25in]


\noindent{Kannan Moudgalya} \\
\noindent{IIT Bombay} \\
\noindent{2 October 2022}