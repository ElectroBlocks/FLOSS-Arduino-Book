\chapter{Introduction}
\thispagestyle{empty}
\label{sec:intro}
Microcontrollers are the foundation for a modern, manufacturing based,
economy.  One cannot fulfill the dreams of one's citizens without a
thriving manufacturing sector.  As it is open source, Arduino is of
particular interest to hobbyists, students, small and medium scale
manufacturers, and people from developing countries, in particular.

Scilab is a state of the art computing software.  It is also open
source.  As a result, this is also extremely useful to the groups
mentioned above.  If the French National Space Agency CNES can
extensively use Scilab \cite{CNES-Scilab}, why can't others rely on
it?  If many of India's satellites can be placed in their precise
orbits by the Ariane rockets launched by CNES through Scilab
calculations, why can't others use Scilab?

Although Arduino and Scilab are versatile, powerful and free, there
has not been much literature that teaches how to integrate them.  To
address this gap, we have written this book.  Xcos is a GUI based
system building tool for Scilab, somewhat similar to
Simulink$^{\textregistered}$\footnote{Simulink$^{\textregistered}$ is
  a registered trademark of Mathworks, Inc.}.  Through Xcos, it is
possible to build interconnected systems graphically.  Xcos also is an
open source software tool.  In this book, we provide Xcos code to
drive \arduino\ board.

The only way we can become versatile in hardware is through hands-on
training.  To this end, we make use of the easily available low cost
\arduino\ board to introduce the reader to computer interfacing.  We
also make available the details of a shield that makes the Arduino use
extremely easy and intuitive.  We tell the user how to install the
firmware to make the \arduino\ board communicate with the computer.
We explain how to control the peripherals on the \arduino\ board with
user developed software.

The Scilab Arduino toolbox is already available for Windows
\cite{scilab-arduino}.  We have suitably modified it, so that it works
on Linux also.  In addition to these toolboxes, we provide the
firmware and a program to check it.  Finally, we give the required
programs to experiment with the sensors and actuators that come with
the shield, a DC motor and a servomotor.  These programs are available
for all of the following three environments: Arduino IDE, Scilab
scripts and Xcos.

This book teaches how to access the following sensors and actuators:
LED, pushbutton, DC motor, Potentiometer and Servo motor.  A set of
two to five programs are given for each.  These are given for Arduino
IDE, Scilab and Xcos.  We explain where to find these programs and how
to execute them for each experiment.

This book is written for self learners and hobbyists.  It has been
field tested by 250 people who attended a hands-on workshop conducted
at IIT Bombay in July 2015.  It has also been field tested by 25
people who participated in a TEQIP course held in Amravati in November
2015.  

All the code described in this book is available at
\url{http://os-hardware.in/arduino/scilab-arduino-files.zip}.  On
downloading and unzipping it, it will open a folder {\tt Origin} in
the current directory.  All the files mentioned in this book are
with reference to this folder\footnote{\label{fn:file-loc}This naming
  convention will be used throughout this book.  Users are expected to
  download this file and use it while reading this book.}.
