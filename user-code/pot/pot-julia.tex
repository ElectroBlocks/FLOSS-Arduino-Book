\section{Reading the potentiometer from Julia}
\subsection{Reading the potentiometer}
In this section, we will use a Julia source file to read 
the potentiometer values. The shield has to be attached to the \arduino\ board
before doing these experiments and the \arduino\ needs to be 
connected to the computer with a USB cable, 
as shown in \figref{arduino}. The reader should go through the instructions given in 
\secref{sec:julia-start} before getting started.


Based on the acquired potentiometer values, we will change the 
state of the RGB LED as explained earlier. 
The code for this experiment is given in
\juliaref{julia:pot-100}. As explained earlier in \secref{sec:light-julia}, we begin with importing the SerialPorts 
\cite{julia-serial-ports} package and the module ArduinoTools followed by setting up the serial port.
Then, we read the analog input at pin 2 using,
\lstinputlisting[firstline=9,lastline=9]
{\LocPotjuliacode/pot-threshold.jl} where the first argument is for
%\redcolor {the kit number } 
the serial port and the second argument corresponds to the analog pin 
to be read.  Next, we compare the read values with the set range, 
and then turn on and off the corresponding LED. For example, 
\lstinputlisting[firstline=11,lastline=14]
{\LocPotjuliacode/pot-threshold.jl} where {\tt digiWrite} 
is used to set the pin 11 high (1) or low (0). 
We used {\tt sleep(1)} to retain the LED in the on state for 1 second.  
A similar check is done the other two bands. 
While running this experiment, 
the readers must rotate the knob of the potentiometer and observe 
the change in the color of the RGB LED.
\subsection{Julia Code}
\label{sec:pot-julia-code}
\addtocontents{juliad}{\protect\addvspace{\codclr}}
\begin{juliacode}
  \jcaption{Turning on LEDs depending on the potentiometer
    threshold}{Turning on LEDs depending on the potentiometer
    threshold.  Available at
  \LocPotjuliabrief{pot-threshold.jl}.}
\label{julia:pot-100}
\lstinputlisting{\LocPotjuliacode/pot-threshold.jl}
\end{juliacode}

