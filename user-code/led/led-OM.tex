\section{Lighting the LED from OpenModelica}
\subsection{Lighting the LED}
\label{sec:light-OpenModelica}
In this section, we discuss how to carry out the experiments of the
previous section from OpenModelica.  We will list the same four experiments,
in the same order.  The Shield has to be attached to the \arduino\ board
before doing these experiments and the \arduino\ needs to be connected to the computer 
with a USB cable, as shown in \figref{arduino}.
The reader should go through the instructions given in
\secref{sec:OpenModelica-start} before getting started.

\begin{enumerate}
  \item In the first experiment, we will light up the blue LED on the
        Shield.  The code for this is given in \OpenModelicaref{OpenModelica:led-blue}.
        It begins with importing the two packages: Streams and SerialCommunication from the toolbox, as 
        given in \secref{sec:load-om-toolbox}. Following line imports this package: 
        \lstinputlisting[firstline=3,lastline=4]{\LocLEDOpenModelicacode/led-blue.mo}

    We define some variables to collect the results coming from different functions. Following 
    lines are used for these:
    \lstinputlisting[firstline=5,lastline=7]{\LocLEDOpenModelicacode/led-blue.mo}

    Now, we have a command of the form 
        \begin{lstlisting}[style=nonumbers]
     ok := sComm.open_serial(1, PORT NUMBER, BAUD RATE)
  \end{lstlisting}
        We have used 2 for {\tt PORT NUMBER} and 115200 for {\tt BAUD RATE}.
        As a result, this command becomes 
        \lstinputlisting[firstline=10,lastline=10]{\LocLEDOpenModelicacode/led-blue.mo}
        This command is used to open the serial port.  When the port is
        opened successfully, it returns a value of 0, which gets stored in
        the variable {\tt ok}.
        
        Sometimes, the serial port does not open, as mentioned in the above
        command.  This is typically due to not closing the serial port
        properly in a previous experiment.  If this condition is not
        trapped, the program will wait forever, without any information
        about this difficulty.  One way to address this difficulty is to
        terminate the program if the serial port does not open.  This is
        achieved using the error message of the following form:
        \begin{lstlisting}[style=nonumbers]
            if ok <> 0 then strm.print(Error Message in Quotes);
        \end{lstlisting}

        We turn the LED on in the upcoming lines.  This is achieved using a
        command of the form
        \begin{lstlisting}[style=nonumbers]
          digital_out := sComm.cmd_digital_out(1, PIN NUMBER, VALUE)
  \end{lstlisting}
        As we want to turn on the blue light in the Shield, as discussed in
        \secref{sec:light-ard}, we choose {\tt PIN NUMBER} as 9.  We can put
        any positive integer in the place of {\tt VALUE}.  We arrive at the
        following command:
        \lstinputlisting[firstline=15,lastline=15]{\LocLEDOpenModelicacode/led-blue.mo}
        Subsequently, we close the serial port and then define the simulation parameters. 
  \item \OpenModelicaref{OpenModelica:led-blue-delay} does the same thing as what
        \ardref{ard:led-blue-delay} does.  It does two more things than what
        \OpenModelicaref{OpenModelica:led-blue} does: It makes the blue LED light up for two
        seconds.  This is achieved by the command
        \lstinputlisting[firstline=16,lastline=16]{\LocLEDOpenModelicacode/led-blue-delay.mo}
        The second thing this code does is to turn the blue LED off.  This
        is achieved by the command
        \lstinputlisting[firstline=17,lastline=17]{\LocLEDOpenModelicacode/led-blue-delay.mo}
        It is easy to see that this code puts a 0 on pin 9.
        
  \item \OpenModelicaref{OpenModelica:led-blue-red} does the same thing as what
        \ardref{ard:led-blue-red} does.  It turns blue and red LEDs on for
        five seconds.  After that, it turns off blue first.  After 3
        seconds, it turns off red also.  So, when the program ends, no LED is
        lit up.
        
  \item \OpenModelicaref{OpenModelica:led-green-blink} does exactly what its counterpart
        in the Arduino IDE does.  It makes the green LED blink five times.
  \end{enumerate}

\subsection{OpenModelica Code}
Unlike other code files, the code/ model for running experiments using OpenModelica are 
available inside OpenModelica-Arduino toolbox, as explained in \secref{sec:load-om-toolbox}.
Please refer to \figref{om-examples-toolbox} to know how to locate the experiments. 
\lstset{style=mystyle}
\label{sec:led-OpenModelica-code}
\addtocontents{OpenModelicad}{\protect\addvspace{\codclr}}

\begin{OpenModelicacode}
  \mcaption{Turning on the blue LED}
  {Turning on the blue LED. Available at Arduino -> SerialCommunication -> 
  Examples -> led -> led\_blue.}
  \label{OpenModelica:led-blue}
  \lstinputlisting{\LocLEDOpenModelicacode/led-blue.mo}
\end{OpenModelicacode}

\begin{OpenModelicacode}
  \mcaption{Turning on the blue LED and turning it off after two
    seconds}{Turning on the blue LED and turning it off after two
    seconds.  Available at Arduino -> SerialCommunication -> Examples -> led 
    -> led\_blue\_delay.}
  \label{OpenModelica:led-blue-delay}
  \lstinputlisting{\LocLEDOpenModelicacode/led-blue-delay.mo}
\end{OpenModelicacode}

\begin{OpenModelicacode}
  \mcaption{Turning on blue and red LEDs for 5 seconds and then turning
    them off one by one}{Turning on blue and red LEDs for 5 seconds and
    then turning them off one by one.  Available at Arduino -> SerialCommunication -> Examples -> led -> led\_blue\_red.}
  \label{OpenModelica:led-blue-red}
  \lstinputlisting{\LocLEDOpenModelicacode/led-blue-red.mo}
\end{OpenModelicacode}

\begin{OpenModelicacode}
  \mcaption{Blinking the green LED}{Blinking the green LED.  Available at Arduino -> SerialCommunication -> Examples -> led -> led\_green\_blink.}
  \label{OpenModelica:led-green-blink}
  \lstinputlisting{\LocLEDOpenModelicacode/led-green-blink.mo}
\end{OpenModelicacode}

%%%%%end OpenModelicamo
