\section{Controlling servomotor through OpenModelica}
\subsection{Controlling the servomotor}
\label{sec:servo-OpenModelica}
In this section, we discuss how to carry out the experiments of the
previous section from OpenModelica.  We will list the same four experiments,
in the same order. The Shield has to be attached to the \arduino\ board
before doing these experiments and the \arduino\ needs to be connected to the computer 
with a USB cable, as shown in \figref{arduino}. The reader should go through the instructions given in
\secref{sec:OpenModelica-start} before getting started.

\begin{enumerate}

  \item The first experiment makes the servomotor move by $30^\circ$. The code for this experiment is
        given in \OpenModelicaref{OpenModelica:servo-init}. As explained earlier in \secref{sec:light-OpenModelica}, 
        we begin with importing the two packages: Streams and SerialCommunication followed 
        by setting up the serial port. Next, we attach the servomotor by issuing the command given below:
        
        \lstinputlisting[firstline=13,lastline=13]{\LocSEROpenModelicacode/servo-init.mo}
        As shown above, the servomotor is attached on board 1 (the first argument)
        to pin 1 (the second argument).  In OpenModelica-Arduino toolbox discussed 
        in \secref{sec:load-om-toolbox}, pin 1 and pin 5 are connected. As a result, we connect the wire physically to
        pin 5, which is achieved by the Shield as discussed in \secref{sec:servo-pril}.
        
        With this, we issue the command to move the servomotor by $30^\circ$ followed by a delay of 
        1 second:
        \lstinputlisting[firstline=14,lastline=14]
        {\LocSEROpenModelicacode/servo-init.mo}
        At last, we  detach the servomotor followed by closing the serial port. 
        
        Once this code is executed, the servomotor would move by
        $30^\circ$, as commanded.  What happens if this code is executed
        once again?  The motor will not move at all.  What is the reason?
        Recall that what we assign to the motor are absolute positions, with
        respect to a fixed origin.  As a result, there will be no change at
        all. 
        
  \item In the second experiment, we move the servomotor by $90^\circ$ in the
        forward direction and $45^\circ$ in the reverse direction.  This
        code is given in \OpenModelicaref{OpenModelica:servo-reverse}.  As mentioned
        earlier, the angles are absolute with respect to a fixed reference
        point and not relative.   In this code, 
        we have added a delay of 1000 milliseconds between the two instances of 
        moving the servomotor: 
        \lstinputlisting[firstline=15,lastline=17]
        {\LocSEROpenModelicacode/servo-reverse.mo}
        What is the reason behind this delay?  If the delay were not
        there, the motor will move only by the net angle of $90-45 = 45$
        degrees.  The reader should verify this by commenting on the delay
        command. 
        
        
  \item In the third experiment, we move the motor in increments of
        $20^\circ$.  This is achieved by the {\tt for} loop, as in
        \OpenModelicaref{OpenModelica:servo-loop}. The code helps the motor move in steps of $20^\circ$ all
        the way to $180^\circ$.  
        
  \item Finally, in the last experiment, we read the potentiometer value
        from the Shield and use it to drive the servomotor, see
        \OpenModelicaref{OpenModelica:servo-pot}.  The resistance of the potentiometer is
        represented in 10 bits.  As a result, the resistance value could be
        any one of 1024 values, from 0 to 1023.  This entire range is
        mapped to $180^\circ$, as shown below:
        \lstinputlisting[firstline=17,lastline=19]
        {\LocSEROpenModelicacode/servo-pot.mo}
        By rotating the potentiometer, one can make
        the motor move by different amounts.
        
        As mentioned in \chapref{potmeter}, the potentiometer on the Shield is connected 
        to analog pin 2 on \arduino. Through this pin, the resistance of the potentiometer, in the range of 0 to 1023,
        depending on its position, is read.  Thus, by rotating the
        potentiometer, we make different values appear on pin 2.  This value
        is used to move the servo.  For example, if the resistance is half
        of the total, the servomotor will go to $90^\circ$ and so on.  The
        servomotor stops for 500 milliseconds after every move.  The loop is
        executed for 50 iterations. During this period, the servomotor keeps moving as dictated by the
        resistance of the potentiometer. While running this experiment, the readers 
        must rotate the knob of the potentiometer by a fixed amount and observe 
        the change in the position (or angle) of the servomotor. 
        
\end{enumerate}

\subsection{OpenModelica Code}
\lstset{style=mystyle}
\label{sec:servo-OpenModelica-code}
Unlike other code files, the code/model for running experiments using OpenModelica are 
available inside OpenModelica-Arduino toolbox, as explained in \secref{sec:load-om-toolbox}.
Please refer to \figref{om-examples-toolbox} to know how to locate the experiments. 

\addtocontents{OpenModelicad}{\protect\addvspace{\codclr}}

\begin{OpenModelicacode}
  \mcaption{Rotating the servomotor to a specified degree} {Rotating
    the servomotor to a specified degree. Available at Arduino -> 
    SerialCommunication -> Examples -> servo -> servo\_init.}
  \label{OpenModelica:servo-init}
  \lstinputlisting{\LocSEROpenModelicacode/servo-init.mo}
\end{OpenModelicacode}

\begin{OpenModelicacode}
  \mcaption{Rotating the servomotor to a specified degree and
    reversing} {Rotating
    the servomotor to a specified degree and reversing.  Available at Arduino -> 
    SerialCommunication -> Examples -> servo -> servo\_reverse.}
  \label{OpenModelica:servo-reverse}
  \lstinputlisting{\LocSEROpenModelicacode/servo-reverse.mo}
\end{OpenModelicacode}

\begin{OpenModelicacode}
  \mcaption{Rotating the servomotor in steps of $20^\circ$}{Rotating
    the servomotor in steps of $20^\circ$.  Available at Arduino -> 
    SerialCommunication -> Examples -> servo -> servo\_loop.}
  \label{OpenModelica:servo-loop}
  \lstinputlisting{\LocSEROpenModelicacode/servo-loop.mo}
\end{OpenModelicacode}

\begin{OpenModelicacode}
  \mcaption{Rotating the servomotor to a degree specified by the
    potentiometer} {Rotating the servomotor to a degree specified by
    the potentiometer.  Available at Arduino -> 
    SerialCommunication -> Examples -> servo -> servo\_pot.}
  \label{OpenModelica:servo-pot}
  \lstinputlisting{\LocSEROpenModelicacode/servo-pot.mo}
\end{OpenModelicacode}
